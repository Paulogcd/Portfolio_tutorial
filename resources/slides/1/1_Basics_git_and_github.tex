\documentclass{beamer}
\usetheme{metropolis}

\title{Basics of Git and GitHub}
\author{Paulo Gugelmo Cavalheiro Dias}

\begin{document}

\section{Introduction}

\begin{frame}
    What is internet ? 

    Internet is a network of computers that can communicate with each other.

    Simply understood, a website is a series of interconnected pages that can be accessed over the internet.

    This tutorial aims to let you create, manage a simple website (the content) and publish it (making it accessible on internet). 
\end{frame}

\begin{frame}
    To create a basic website, one should usually know some HTML and CSS, the two main languages of web development.

    In last years, it has become possible to create a website with static website generators, like \hyperlink{https://jekyllrb.com}{Jekyll}, \hyperlink{https://www.getzola.org/}{Zola}, \hyperlink{https://www.gohugo.io/}{Hugo}, or \hyperlink{http://quarto.org}{Quarto}.

    These are tools that makes it the conception and maintenance of websites easy and quick. This tutorial focuses on Quarto.

    To use Quarto, we will first need to use Visual Studio Code and Git.
\end{frame}

\begin{frame}
    First step : Install Visual Studio Code.
\end{frame}

\begin{frame}
    Second step : Install Git.
\end{frame}

\begin{frame}
    Understand how Git works.
\end{frame}

\begin{frame}
    Give Git another dimension : Git and Github together. Explain what they do together. 
\end{frame}

\begin{frame}
    Now, create a Github account, create a repository, and follow the steps. 
\end{frame}

\begin{frame}
    Now, you can synchronizing your local directory to the remote on, hosted by GitHub.
\end{frame}

\begin{frame}
    The triforce : add, commit, and push.
\end{frame}

\begin{frame}
    And it will be connected to your distant folder on GitHub ! 
\end{frame}

\end{document}